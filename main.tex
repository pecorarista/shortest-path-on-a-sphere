\documentclass{ltjsarticle}
% \usepackage[%
%     textwidth=40\zw,
%     lines=40,
%     centering
% ]{geometry}
\usepackage{graphicx}
\usepackage{subcaption}
\usepackage[%
    hang,
    flushmargin
]{footmisc}
\usepackage{amsmath}
\usepackage{amssymb}
\usepackage{amsthm}
\usepackage{cases}
\usepackage[
    math-style=ISO
]{unicode-math}
\usepackage{tikz}
\usepackage{tikz-3dplot}
\usepackage{tikz-qtree}
\usepackage{pgfplots}
\usetikzlibrary{%
    patterns,
    intersections,
    calc,
    angles,
    quotes
}

\usepackage{xcolor}
\definecolor{sBlue}{HTML}{0095D9}
\colorlet{sOriginalCurve}{sBlue}

\definecolor{sRed}{HTML}{FF0033}
\colorlet{sApproxCurve}{sRed}

\definecolor{sGreen}{HTML}{138D75}
\colorlet{sSimpleCurve}{sGreen}

\definecolor{sFill}{HTML}{00A381}
\definecolor{sLightGray}{gray}{0.85}

\usepackage[
    % no-math
]{fontspec}
\usepackage[
    deluxe,%
    expert,
    scale=0.95
]{luatexja-preset}

\usepackage{luatexja-ruby}
\setmainfont[%
    Extension=.ttf,
    UprightFont=*,
    ItalicFont=*i,
    BoldFont=*bd,
    BoldItalicFont=*bi,
    Ligatures=Discretionary
]{times}
\setsansfont[
    Extension=.otf,
    UprightFont=*-Light,
    BoldFont=*-Roman,
]{FrutigerLTStd}
\setmathfont{NewComputerModernMath-Book}
\setmathfont[%
    range={%
        cal,
        bfcal
    },
    StylisticSet=1
]{KpMath-Regular.otf}

\ltjnewpreset{book}{%
    mc-m = A-OTF-RyuminPro-Regular.otf,
    % mc-m = NotoSerifJP-Regular.otf,
    % mc-m = HiraMinProN-W3.otf,
    gt-m = BIZUDGothic-Regular.ttf,
    gt-b = BIZUDGothic-Bold.ttf,
    mg-m = MotoyaMaruStd-W5.otf
}
\ltjapplypreset{book}
\setmonojfont{HiraMaruProN-W4}
\ltjsetparameter{%
    jacharrange={%
        -2, % Exclude Greek and Cyrillic letters.
        -3, % Punctuations and miscellaneous symbols.
        -7  % other CJK characters
    },
    alxspmode={`/,allow},
    alxspmode={`#,allow},
    alxspmode={92,allow} % backslash
}

\newtheoremstyle{jplain}%
    {\topsep}%
    {\topsep}%
    {\normalfont}%
    {}%
    {\bfseries\gtfamily\sffamily}%
    {}%
    {5pt}%
    {\thmname{#1}\thmnumber{#2}\thmnote{#3}\ignorespaces}
\newtheoremstyle{jspecial}%
    {\topsep}%
    {\topsep}%
    {\normalfont}%
    {}%
    {\bfseries\gtfamily\sffamily}%
    {}%
    {5pt}%
    {\thmname{#1}\thmnumber{#2\({}^{\symbf{*}}\)}\thmnote{#3}\ignorespaces}
\theoremstyle{jplain}
\newtheorem{theorem}{定理}[section]
\newtheorem{proposition}[theorem]{命題}
\newtheorem{lemma}[theorem]{補題}
\newtheorem{definition}[theorem]{定義}
\newtheorem{example}[theorem]{例}
\newtheorem{axiom}[theorem]{公理}
\newtheorem{corollary}[theorem]{系}
\newtheorem{numberedquote}[theorem]{引用}
\renewcommand{\proofname}{証明}
\renewcommand{\qedsymbol}{\rule{5pt}{10pt}}
\theoremstyle{jspecial}
\newtheorem{specialexample}[theorem]{例}

\usepackage{wtref}
\newref{fig}
\setrefstyle{fig}{prefix=図}
\newref{math}
\setrefstyle{math}{prefix=式}
\newref{definition}
\setrefstyle{definition}{prefix=定義}
\newref{axiom}
\setrefstyle{axiom}{prefix=公理}
\newref{proposition}
\setrefstyle{proposition}{prefix=命題}
\newref{lemma}
\setrefstyle{lemma}{prefix=補題}
\newref{example}
\setrefstyle{example}{prefix=例}
\newref{theorem}
\setrefstyle{theorem}{prefix=定理}
\newref{chapter}
\setrefstyle{chapter}{prefix={第},suffix={章}}
\newref{sec}
\setrefstyle{sec}{prefix={第},suffix={節}}
\newref{corollary}
\setrefstyle{corollary}{prefix=系}

\usepackage{tcolorbox}
\tcbuselibrary{%
    skins,
    breakable
}
\newtcolorbox{thmbox}{%
    colback=sLightGray,
    before skip=1.3em,
    top=8pt,
    bottom=8pt,
    left=8pt,
    right=8pt,
    boxrule=0pt,
    sharp corners,
    frame hidden
}
\newtcolorbox{quotebox}{%
    breakable,
    colback=sLightGray,
    top=8pt,
    bottom=8pt,
    left=8pt,
    right=8pt,
    boxrule=0pt,
    sharp corners,
    frame hidden
}

\makeatletter
\renewenvironment{proof}[1][\proofname]{\par
    \pushQED{\qed}%
    \normalfont \topsep=1.3em\relax
    \trivlist
    \item[\hskip\labelsep\bfseries\gtfamily#1]\ignorespaces
}{%
    \popQED\endtrivlist\@endpefalse\addvspace{1.3em}
}
\makeatother
\AtBeginDocument{%
    \setlength{\abovedisplayskip}{5pt}%
    \setlength{\belowdisplayskip}{5pt}%
}

% https://note.com/yuw/n/n38dd54fb2169
\begingroup
\catcode`\,=\active
\def\@x@{\def,{\normalcomma\hskip.2em}}
\expandafter\endgroup\@x@%
\mathcode`\,="8000
\def\normalcomma{\mathchar"613B}

\usepackage{enumitem}
\newlist{conditions}{enumerate}{1}
\setlist[conditions]{label=(\arabic*)}

\newlist{inproofconditions}{enumerate}{1}
\setlist[inproofconditions]{label=(\alph*)}

\usepackage{hyperref}
\hypersetup{%
    colorlinks,
    citecolor=sBlue,
    linkcolor=sBlue,
    urlcolor=sBlue
}

\usepackage[%
    backend=biber,
    style=pecorarista,
    sorting=nyvt,
    urldate=long
]{biblatex}
\addbibresource{references.bib}

\newcommand\mainchapter[1]{\chapter{#1}\thispagestyle{empty}}
\renewcommand{\jsParagraphMark}{}
\renewcommand{\headfont}{\bfseries\gtfamily\sffamily}

\newcommand\tuple[1]{(#1)}
\newcommand\coord[1]{(#1)}
\newcommand\openinterval[1]{(#1)}
\newcommand\absolute[1]{\lvert #1 \rvert}
\newcommand\keyword[1]{{\bfseries\gtfamily #1}}
\newcommand\emphchar[1]{\ltjalchar`‹\thinspace{#1}\thinspace\ltjalchar`›}

\newcommand\NaturalNumber{\symbb{N}}
\newcommand\Integer{\symbb{Z}}
\newcommand\Rational{\symbb{Q}}
\newcommand\Real{\symbb{R}}
\newcommand\Complex{\symbb{C}}
\newcommand\PositiveInteger{\symbb{N}_{\mathord{+}}}
\newcommand\NonNegativeReal{\symbb{R}_{\mathord{+}}}
\newcommand\PositiveReal{\symbb{R}_{\mathord{+}\mathord{+}}}

\newcommand\SequenceLikeOpen{\langle}
\newcommand\SequenceLikeClose{\rangle}
\newcommand\Sequence[2]{{\SequenceLikeOpen {#1}_{#2} \SequenceLikeClose}_{#2 \in \PositiveInteger}}
\newcommand\Sequencen[1]{{\SequenceLikeOpen {#1}_n \SequenceLikeClose}_{n \in \PositiveInteger}}
\newcommand\Subsequence[1]{{\SequenceLikeOpen {#1}_{\varphi(n)} \SequenceLikeClose}_{n \in \PositiveInteger}}
\newcommand\Family[3]{{\SequenceLikeOpen {#1}_{#2} \SequenceLikeClose}_{#2 \in #3}}

\newcommand\FamilyLambda[1]{{\SequenceLikeOpen {#1}_\lambda \SequenceLikeClose}_{\lambda \in \Lambda}}

\newcommand\Generatedtopology[1]{\tau\left[#1\right]}
\newcommand\Neighborhood[1]{\symcal{N}_{#1}}
\newcommand\OpenNeighborhood[1]{\symcal{U}_{#1}}

\newcommand\placeholder{\mathord{\text{\char"25CC}}}
\newcommand\powerset[1]{\wp(#1)}
\newcommand\inlinefrac[2]{#1\mathbin{/}#2}
\newcommand\setcomp[1]{{#1}^{\mathsf{c}}}
% https://zrbabbler.hatenablog.com/entry/20120411/1334151482
\newcommand{\relmiddle}[1]{\mathrel{}\middle#1\mathrel{}}

\DeclareMathOperator{\identity}{id}
\DeclareMathOperator{\trace}{trace}
\DeclareMathOperator{\len}{length}
\DeclareMathOperator{\sgn}{sgn}
\DeclareMathOperator{\atantwo}{atan2}
\DeclareMathOperator{\realpart}{Re}
\DeclareMathOperator{\imaginarypart}{Im}
\DeclareMathOperator{\Span}{span}
\newcommand\symdiffsymbol{\mathord{\triangle}}
\newcommand\symdiff[2]{#1\mathbin{\triangle}#2}
\newcommand\formallang[1]{{\treefont\itshape #1}}

\newcommand\header[1]{\multicolumn{1}{c}{\bfseries\gtfamily #1}}
\newcommand\pronunciation[4]{#1 \texttt{#2} [#3] #4.}
\newcommand\primarystress{\mbox{}\char"02C8}
\newcommand\projection[1]{\pi_{#1}}
\renewcommand{\restriction}{\mathrel{\upharpoonright}}

\newcommand\submin[1]{#1_{\text{min}}}

\newcommand\liaison{\hspace*{0.1em}\raisebox{-0.8ex}{\rotatebox{90}{(}}\hspace*{0.1em}}
\newcommand\shortunderscore{\scalebox{0.7}[1]{\_}}
\newcommand\invbreve[1]{#1{\char"032F}}
% \newcommand\symbb[1]{\mathbb{#1}}
% \newcommand\symcal[1]{\mathcal{#1}}
% \newcommand\symbf[1]{\mathbf{#1}}

\newcommand\iseventuallyin[2]{#1 \mathrel{\text{is eventually in}} #2}
\newcommand\justin[1]{\mathrel{\text{in}} #1}
\newcommand\isfrequentlyin[2]{#1 \mathrel{\text{is frequently in}} #2}
\newcommand\oxfordcomma[3]{#1, #2, \text{and }#3}
\newcommand\transpose{\symsfup{T}}
\newcommand\inlinevec[3]{(#1\ \ #2\ \ #3)^{\transpose}}
\newcommand\origin{\symup{O}}
\newcommand\almosteverywhere{\mathrm{a.e.}}

\newcommand\refimpliesref[2]{\(\text{#1} \Rightarrow \text{#2}\)}
\newcommand\tpmbase[2]{\left.\frac{\partial}{\partial #1}\right\lvert_{#2}}
\newcommand\tpRnbase[1]{\frac{\partial}{\partial #1}}

\renewcommand{\labelitemii}{\(\circ\)}
\renewcommand{\labelitemiii}{\(\diamond\)}

\usepackage{natbib}

\pgfplotsset{%
    defaultplot/.style={%
        width=\textwidth,
        height=1.05\textwidth,
        x=1cm,
        y=1cm,
        axis lines=middle,
        axis line style=thick,
        xlabel={},
        ylabel={},
        xtick=\empty,
        ytick=\empty,
        xmin=-0.25,
        xmax=5,
        ymin=-0.25,
        ymax=4.5,
        samples=200,
        clip=false
    }
}

\title{地球上の2点間の距離の求め方}
\author{宮澤 彬}
\begin{document}

\maketitle

\section{Banach空間における微分}

\nocite{precup} % Chapter 7
\nocite{suhubi} % Chapter 9
\begin{thmbox}
\begin{definition}
\(X, Y\)を実Banach空間とし,\(U\)をその開集合とする.
関数\(f \colon U \to Y\)が\(x \in U\)において\(h \in X\)方向に微分可能であるとは,任意の\(\varepsilon > 0\)について,ある\(\delta > 0\)が存在して,\(\lvert s \rvert \leq \delta\)ならば
\begin{align}
    \lVert f(x + s h) - f(x) - s g(x, h) \rVert_Y \leq \varepsilon \lvert s \rvert \mathlabel{directional-derivative}
\end{align}
となるような\(g(x, h)\)が存在することをいう.
任意の\(h \in X\)について\(g(x, h)\)が存在するとき,
\(f\)は\(x\)において\keyword{Gateaux微分可能}(Gateaux differentiable)であるという.
このとき\(g(x, \placeholder) \colon X \to {\Real}\)を\(f'(x)\)と表し,\(f\)の\(x\)における\keyword{Gateaux導関数}(Gateaux derivative)という.
\end{definition}
\end{thmbox}

Gateaux導関数は\(1\)次同次である.すなわち\(f'(x)(\lambda h) = \lambda f' (x)(h)\)が成り立つ.
これは\mathref{directional-derivative}が成り立つとき,\(\lvert s \lambda \rvert \leq \delta_0\)となるように\(s\)をとれば\(\lVert f(x + s \lambda h) - f(x) - s \lambda g(x, h) \rVert_Y \leq \varepsilon \lvert \lambda t \rvert\)
となることからわかる.

一方で,通常の微分と異なり,加法性\(f'(x)(u + v) = f'(x)(u) + f'(q)(v)\)は一般には成り立たない.
例えば\(f \colon {\Real}^2 \mapsto {\Real}\)を
\begin{numcases}
    {f \colon (x_1, x_2) \mapsto}
    0                               & if \(\coord{x_1, x_2} = \coord{0, 0}\), \nonumber \\
    \frac{x_1 x_2^2}{x_1^2 + x_2^2} & otherwise \nonumber
\end{numcases}
と定義すると
\begin{gather*}
    f' (0, 0) (h_1, h_2) = \frac{h_1 h_2^2}{h_1^2 + h_2^2}
\end{gather*}
であり,\(u = \coord{1, 0}, v = \coord{0, 1}\)について
\begin{gather*}
    f' (0, 0) (u + v) =
    \frac{(1 + 0) (0 + 1)^2}{(0 + 1)^2 + (1 + 0)^2} = \frac{1}{2}, \\
    f' (0, 0) (u) + f' (0)(v) = \frac{1 \cdot 0^2}{1^2 + 0^2}
    +
    \frac{0 \cdot 1^2}{0^2 + 1^2} = 0
\end{gather*}
となることからわかる.


\begin{thmbox}
\begin{definition}
\(X, Y\)を実Banach空間とし,\(U\)をその開集合とする.
関数\(f \colon U \to Y\)が\(x \in U\)において\keyword{Fréchet微分可能}(Fréchet differentiable)であるとは,任意の\(\varepsilon > 0\)について,ある\(\delta > 0\)が存在して,\(\lVert h \rVert_X \leq \delta\)ならば
\begin{align}
    \lVert f(x + h) - f(x) - Df(x) h \rVert_Y \leq \varepsilon \lVert h \rVert_X \mathlabel{frechet-derivative}
\end{align}
となるような連続な線型写像\(Df(x) \colon U \to Y\)が存在することをいう.
この\(Df(x)\)を\(f\)の\(x\)における\keyword{Fréchet導関数}(Fréchet derivative)という.
\end{definition}
\end{thmbox}

% Postmodern Analysis p. 108
\nocite{postmodern-analysis}
\({\Real}^n\)から\({\Real}^m\)への写像と同様に,Banach空間からBanach空間への写像についても偏微分を定義することができる.
\(\tuple{X_1, \lVert \placeholder \rVert_{X_1}}, \ldots, \tuple{X_n, \lVert \placeholder \rVert_{X_n}}\)をBanach空間とし,\(X\)をそれらの直積空間\(\tuple{X_1 \times \cdots \times X_n, \lVert \placeholder \rVert_{X_1} + \cdots + \lVert \placeholder \rVert_{X_n}}\)とする.
\(f \colon X \to Y\)が\(x \in X\)において第\(j\)成分に関して\keyword{偏Fréchet微分可能}(partially Fréchet differentiable)あるいは単に偏微分可能であるとは,ある有界線型作用素\(A(x) \colon X_j \to {\Real}\)が存在し,任意の\(\varepsilon > 0\)について,ある\(\delta > 0\)が存在して,\(h_j \in X_j\)が\(\lVert h_j \rVert_{X_j} \leq \delta\)をみたすならば
\begin{align}
    \lVert f(x_1, \ldots, x_j + h_j, \ldots, x_n) - f(x) - A(x) h_j \rVert_{Y} \leq \varepsilon \lVert h_j \rVert_{X_j}
    \mathlabel{partial-frechet-derivative}
\end{align}
ことをいう.このとき\(A(x)\)を\(f\)の第\(j\)成分に関する\keyword{偏Fréchet導関数}(partial Fréchet derivative)あるいは偏導関数といい,\(D_jf(x)\)と表す.
\(f\)が\(X\)において(Fréchet)微分可能ならば,第\(j\)成分に関して偏微分可能である.
このことは次のように確かめられる.
写像\(\iota_j \colon V_j \to V_1 \times \cdots \times V_n\)を以下で定める.
\begin{align*}
    \iota_j \colon v \mapsto
    (
        0,
        \ldots,
        \overset{\overset{\scriptstyle j}{\smallsmile}}{v},
        \ldots,
        0
    )
\end{align*}
このとき任意の\(\varepsilon > 0\)について,ある\(\delta > 0\)が存在して,\(\lVert h_j \rVert_{X} \leq \delta\)ならば
\begin{align*}
    & \lvert f(x_1, \ldots, x_j + h_j, \ldots, x_n) - f(x) - (Df(x) \circ \iota_j) h_j \rvert \\
    & \leq \lvert f(x_1, \ldots, x_j + h_j, \ldots, x_n) - f(x) - Df(x) (0, \ldots, h_j, \ldots, 0) \rvert \\
    & \leq \varepsilon \lVert (0, \ldots, h_j, \ldots, 0) \rVert_{X} = \varepsilon \lVert h_j \rVert_{X_j}
\end{align*}
が成り立つ.
したがって各成分に関して偏微分可能であり,偏導関数はそれぞれ\(D_jf(x) = Df(x) \circ \iota_j\)と表すことができる.
このことと\(Df(x)\)の線型性から
\begin{align*}
    Df(x)(h_1, \ldots, h_n)
    = Df(x) \left(\sum_{j = 1}^n (0, \ldots, h_j, \ldots, 0) \right)
    = \sum_{j = 1}^n (Df(x) \circ \iota_j)h_j
    = \sum_{j = 1}^n D_j(x) h_j
\end{align*}
が成り立つ.

\begin{thmbox}
\begin{theorem}
\(U\)を\(X\)の開集合とし,\(f \colon U \to Y\)とする.
\(f\)が\(x \in U\)においてFréchet微分可能ならばGateaux微分可能である.
また\(Df(x)h\)が意味をもつ任意の\(h\)について\(Df(x)h = f'(x)(h)\)が成り立つ.\theoremlabel{frechet-gateaux}
\end{theorem}
\end{thmbox}

\begin{proof}
\(h = 0\)のときは成り立つ.
任意の\(h \in X \setminus \{0\}\)をとる.
\(f\)が\(x\)でFréchet微分可能なので,
任意の\(\varepsilon > 0\)について,ある\(\delta > 0\)が存在して,\(\lVert \eta \rVert_X \leq \delta\)ならば
\begin{align*}
    \lVert f(x + \eta) - f(x) - Df(x) (\eta) \rVert_Y \leq \frac{\varepsilon}{\lVert h \rVert_X} \lVert \eta \rVert_X
\end{align*}
が成り立つ.このとき\(\lvert s \rvert \leq \inlinefrac{\delta}{\lVert h \rVert_X}\)ならば
\begin{align}
    \lVert f(x + s h) - f(x) - s D(x)(h) \rVert_Y \leq \varepsilon \lvert s \rvert
\end{align}
となる.したがって\(\lVert h \rVert \leq \delta\)ならば\(f'(x)(h) = Df(x)h\)が成り立つ.
\end{proof}

\section{曲線の長さ}

\(\tuple{M, g}\)をRiemann多様体とする.
任意の点\(p \in M\)における方向微分\((\inlinefrac{\partial}{\partial x_i})\rvert_p\)を\(\partial_{x_i}\rvert_{p}\)や\(\partial_{x_i}\)のように略記する.
また\(C^1\)級曲線\(\gamma \colon [a, b] \to M\)の時刻\(t = \tau\)における速度ベクトル\(\displaystyle d\gamma((\inlinefrac{\partial}{\partial t})\rvert_{\tau})\)を\(\dot{\gamma}(\tau)\)と略記する.
通常の微積分の意味での\(t\)に関する導関数も同様に\(\dot{\theta}, \dot{\varphi}\)のように表すことがある.
\(C^1\)級曲線\(\gamma \colon [a, b] \to M\)の長さを
\begin{align*}
    L(\gamma) = \int_{a}^{b} \sqrt{g(\dot{\gamma}(\tau), \dot{\gamma}(\tau))} d\tau
\end{align*}
と表す.

\begin{thmbox}
\begin{proposition}
\(\gamma \colon [a, b] \to M\)を\(C^1\)級曲線とする.
\(C^1\)級狭義単調増加関数\(u \colon [k, \ell] \to [a, b]\)について,
\(\tilde{\gamma} := \gamma \circ u\)とすると次が成り立つ.
\begin{align}
    L(\tilde{\gamma}) = L(\gamma) \mathlabel{great-circle-length-invariance}
\end{align}
\end{proposition}
\end{thmbox}

\begin{proof}
曲線上の任意の点\(\gamma(\tau)\)がその近傍の局所座標\(\psi \colon p \mapsto (x_1, \ldots, x_n)\)によって\((\psi \circ \gamma)(\tau) = (\gamma_1(\tau), \ldots, \gamma_n(\tau))\)と表されているとする.
\(\tilde{\gamma} := \gamma \circ u, \tau = u(\sigma)\)とすると
\begin{align*}
    \frac{d\tilde{\gamma}}{ds}(\sigma)
    &= d(\gamma \circ u)(\sigma) \\
    &= d\gamma \left(du \left(\tpmbase{s}{\sigma}\right) \right) \\
    &= d\gamma \left(\frac{du}{ds}(\sigma) \tpmbase{t}{u(\sigma)} \right) \\
    &= \left(\frac{du}{dt}(\tau) \right)^{-1} d\gamma\left(\tpmbase{t}{\tau}\right) \\
    &= \frac{1}{\dot{u}(\tau)} \dot{\gamma}(\tau)
\end{align*}
このとき
\begin{align}
    g_{\tilde{\gamma}(\sigma)}\left(\frac{d\tilde{\gamma}}{ds}(\sigma), \frac{d\tilde{\gamma}}{ds}(\sigma)\right)
    = \frac{1}{(\dot{u}(\tau))^2}
    g_{\gamma(\tau)}\left(\dot{\gamma}(\tau), \dot{\gamma}(\tau)\right).
    \mathlabel{great-circle-unit-speed-metric}
\end{align}
したがって
\begin{align*}
    \int_{k}^{\ell} \sqrt{%
        g_{\tilde{\gamma}(\sigma)}\left(\frac{d\tilde{\gamma}}{ds}(\sigma), \tilde{\gamma}(\sigma)\right)
    } d\sigma
    = \int_{a}^{b} \sqrt{%
        g\left(\dot{\gamma}(\tau), \dot{\gamma}(\tau)\right)
    } d\tau,
\end{align*}
すなわち\mathref{great-circle-length-invariance}が成り立つ.
\end{proof}

曲線\(\gamma \colon [a, b] \to M\)が正則である,すなわち任意の\(\tau \in [a, b]\)で\(\dot{\gamma}(\tau) \neq 0\)が成り立つとする.
\(a\)から\(t\)までの\(\gamma\)の長さを表す関数\(\ell \colon [a, b] \to [0, L(\gamma)]\)は
\begin{align*}
    \ell(t) = \int_{a}^{t} \sqrt{g(\dot{\gamma}(\tau), \dot{\gamma}(\tau))} d \tau
\end{align*}
と表される.正則性の仮定より
\begin{align*}
    \frac{d\ell}{dt}(\tau) = \sqrt{g(\dot{\gamma}(\tau), \dot{\gamma}(\tau))} > 0
\end{align*}
であるから,\(\ell\)の逆関数\(u \colon [0, L(\gamma)] \to [a, b]\)が存在する.\mathref{great-circle-unit-speed-metric}から
\begin{align}
    g\left(\frac{d\tilde{\gamma}}{ds}(\sigma), \frac{d\tilde{\gamma}}{ds}(\sigma)\right) = 1
    \mathlabel{great-circle-unit-speed}
\end{align}
が成り立つ.\(\sqrt{g_p(v, v)}\)をノルム\(\lVert v \rVert_p\)とみなすと,\mathref{great-circle-unit-speed}は速度(速度ベクトルの大きさ)が常に\(1\)であると解釈される.
このような弧長を用いた再パラメーター付け(arc length reparametrization, unit speed reparametrization)について次が成り立つ.

\begin{thmbox}
\begin{proposition}
\(\gamma \colon [a, b] \to M\)を\(C^1\)級曲線とする.
任意の\(t \in [a, b]\)で\(\sqrt{g(\dot{\gamma}(\tau), \dot{\gamma}(\tau))} = \mathrm{const.}\)が成り立つならば,
長さ\(L(\gamma)\)は\keyword{エネルギー}(energy)
\begin{align*}
    E(\gamma) := \int_{a}^{b} \frac{1}{2} g_{\gamma(\tau)}(\dot{\gamma}(\tau), \dot{\gamma}(\tau)) d\tau
\end{align*}
に比例する.
\end{proposition}
\end{thmbox}

\begin{proof}
\(L^2\)空間におけるCauchy--Schwarzの不等式により
\begin{gather}
    \int_{a}^{b} \sqrt{1} \sqrt{g(\dot{\gamma}(t), \dot{\gamma}(t))} dt
    \leq
    \sqrt{\int_a^b 1 dt}
    \sqrt{%
        \int_{a}^{b} g(\dot{\gamma}(t), \dot{\gamma}(t)) dt.
    }
    \mathlabel{great-circle-cauchy-schwarz}
\end{gather}
\(\sqrt{g(\dot{\gamma}(t), \dot{\gamma}(t))} = \mathrm{const.}\)より\mathref{great-circle-cauchy-schwarz}は等号で成立する.したがって\(L(\gamma) = \sqrt{2 (b - a)}\sqrt{E(\gamma)} \)となる.
\end{proof}


\section{変分法}
関数の空間\(X\)から\({\Real}\)への関数を\keyword{汎関数}(functional)という.
汎関数を最大化あるいは最小化するような関数を求める問題を\keyword{変分問題}(variational problem)という.
典型的なものとして,最も早く降りられる滑り台の形状を求める問題や,\(2\)点間の距離を最小とするような曲線を求める問題がある.
この種の問題をここでは次のように定式化する.
\(U\)を\({\Real}^n\)の開集合とし,\(f \colon U \times {\Real}^n \to {\Real}\)を連続関数とする.
関数(曲線)の空間として\(X = C^1([a, b], {\Real}^n)\)をとる.
この空間のノルムは
\begin{align}
    \lVert \gamma \rVert_{C^1([a, b], {\Real}^n)} =
    \sup_{t \in [a, b]} \lVert \gamma(t) \rVert_{{\Real}^n} + \sup_{t \in [a, b]} \lVert \dot{\gamma}(t) \rVert_{{\Real}^n}
    \mathlabel{c1-norm}
\end{align}
である.\(X_{\symup{P}, \symup{Q}} := \{\gamma \in X \mid \gamma(a) = \symup{P}, \gamma(b) = \symup{Q}\}\)と表すとき,次を最小化する\(\gamma \in X_{\symup{P}, \symup{Q}}\)を求めよ.
\begin{align}
    J(\gamma) = \int_a^b f(\gamma(t), \dot{\gamma}(t))dt. \problemlabel{variational-problem}
\end{align}

\(X_0 = \{\gamma \in X \mid \gamma(a) = \gamma(b) = 0\}\)とする.
\(U\)が開集合なので,任意の\(q \in X_{\symup{P}, \symup{Q}}\)と\(\eta \in X_0\)について,
ある\(r > 0\)が存在して,任意の\(t \in [a, b], s \in \openinterval{-r, r}\)で\(q(t) + s \eta(t) \in U\)とできる.
\(q \in X_{\symup{P}, \symup{Q}}\)が\problemref{variational-problem}を最小化するならば,
\begin{align*}
    J(q + s \eta) \geq J(q), \quad J(q + s (- \eta)) \geq J(q),
\end{align*}
が成り立つ.このとき
\begin{align*}
    J' (q)(\eta) \geq 0, \quad - J'(q)(\eta) \geq 0
\end{align*}
が成り立ち,ゆえに\(J'(q) = 0\)となる.

\begin{thmbox}
\begin{proposition}
\(U\)を\({\Real}^n\)の開集合とし,関数\(f \colon U \times {\Real}^n \to {\Real}\)を\(C^1\)級とする.
汎関数\(J \colon X_{\symup{P}, \symup{Q}} \to {\Real}\)を
\begin{align*}
    J(\gamma) = \int_a^b f(\gamma(t), \dot{\gamma}(t)) dt
\end{align*}
と定める.
このとき\(J\)の\(q\)におけるGateaux導関数は
\begin{align}
    J' (q) \colon \eta \mapsto \int_a^b \left( D_1 f(q(t), \dot{q}(t)) \eta(t)  + D_2 f(q(t), \dot{q}(t)) \dot{\eta}(t) \right) dt \mathlabel{variational-method-frechet-derivative}
\end{align}
で与えられる.
\end{proposition}
\end{thmbox}

\begin{proof}
\(U\)が開集合であるから,
任意の\(q \in X_{\symup{P}, \symup{Q}}\)と\(\eta \in X_0 \setminus \{0\}\)について,
ある\(r > 0\)が存在して,任意の\(t \in [a, b], s \in \openinterval{-r, r}\)で\(q(t) + s \eta(t) \in U\)とできる.

\(f\)が微分可能であるから,任意の\(\varepsilon > 0\)について,ある\(\delta > 0\)が存在して,\(\lvert s \rvert \leq \delta\)ならば
\begin{align*}
    \lvert f(x_1 + s h_1, x_2 + s h_2) - f(x_1, x_2) - s D_1 f(x_1, x_2) h_1
    - s D_2 f (x_1, x_2) h_2  \rvert \leq \frac{\varepsilon \lvert s \rvert}{b - a}
\end{align*}
が成り立つ.
したがって\(\lvert s \rvert \in \min \{r, \delta\}\)ならば
\begin{align*}
    &\left\lvert J(q + s \eta) - J(q) - s \int_a^b  (D_1 f(q(t), \dot{q}(t)) \eta(t) + D_2 f (q(t), \dot{q}(t)) \dot{\eta}(t))  dt \right\rvert \\
    &\leq \int_a^b \left\lvert f(q(t) + s \eta(t), \dot{q}(t) + s \dot{\eta}(t)) - f(q(t), \dot{q}(t))
    - s D_1 f(q(t), \dot{q}(t)) \eta(t) - s D_2 f (q(t), \dot{q}(t)) \dot{\eta}(t)) \right\rvert dt \\
    &\leq \int_a^b \frac{\varepsilon \lvert s \rvert}{b - a} = \varepsilon \lvert s \rvert.
\end{align*}
よって\mathref{variational-method-frechet-derivative}が成り立つ.
\end{proof}

\begin{thmbox}
\begin{lemma}
\(f \in C([a, b], {\Real}^n), f \colon t\mapsto \inlinecolvec{f_1(t)}{\cdots}{f_n(t)}\)とする.
任意の連続関数\(\eta \in X_0\)について
\begin{align}
    \int_a^b f(t) \cdot \eta(t) dt = 0
    \label{fundamental-lemma-assumption}
\end{align}
が成り立つならば任意の\(t \in [a, b]\)で\(f(t) = 0\)が成り立つ.\lemmalabel{fundamental-lemma1}
\end{lemma}
\end{thmbox}

\begin{proof}
背理法によって示す.
ある\(j \in \{1, \ldots, n\}\)と\(t_0 \in [a, b]\)で\(f_j(t_0) > 0\)と仮定する.
\(f_j\)が連続であるから,ある\(\varepsilon > 0\)が存在して\(t \in [t_0 - \varepsilon, t_0 + \varepsilon]\)ならば\(f_j(t) > 0\)となる.
\(\eta \colon t \mapsto \inlinecolvec{\eta_1(t)}{\cdots}{\eta_n(t)}\)を
\begin{numcases}
    {\eta_i(t) =}
    \exp \left(\frac{1}{(t - (t_0 - \varepsilon))(t - (t_0 + \varepsilon))} \right) \symbb{1}_{[t_0 -\varepsilon, t_0 + \varepsilon]}(t) & if \(i = j\), \nonumber \\
    0 & otherwise \nonumber
\end{numcases}
と定義する(\figref{bump-function}).
このとき
\begin{align*}
    \int_a^b f(t) \cdot \eta(t) dt = \int_{t_0 - \varepsilon}^{t_0 + \varepsilon} f_j(t) \eta_j(t) > 0
\end{align*}
となり,仮定に矛盾する.
\end{proof}

\begin{figure}
\centering
\begin{tikzpicture}
    \begin{axis}[
        defaultplot,
        xlabel={\(t\)},
        ylabel={\(y\)},
        ymax=2.5,
        xtick={1, 2.25, 3.5},
        xticklabels={\(t_0 - \varepsilon\), \(t_0\), \(t_0 + \varepsilon\)}
    ]
    \node (O) at (axis cs: -0.3, -0.3) {\(\origin\)};
    \addplot[dotted] coordinates {(2.25, 0) (2.25, 1.05)};
    \addplot[sOriginalCurve, thick][domain=1.01:3.49] {2 * exp(((x - 1) * (x - 3.5))^(-1))};
    \addplot[sOriginalCurve, thick] coordinates {(3.489, 0) (4.6, 0)};
    \addplot[sOriginalCurve, thick] coordinates {(-0.15, 0) (1.02, 0)};
\end{axis}
\end{tikzpicture}
\caption{\(y = \eta_j(t)\)のグラフ.これはbump functionと呼ばれるものの一種である.}
\figlabel{bump-function}
\end{figure}

\begin{thmbox}
\begin{theorem}
\(f \in C^1([a, b], {\Real}^{n})\)とする.
任意の\(\eta \in X_0\)について
\begin{align}
    \int_a^b f (t) \cdot \dot{\eta}(t) dt = 0
    \label{fundamental-lemma2-assumption}
\end{align}
が成り立つならば\(f\)は定数関数である.\theoremlabel{fundamental-lemma2}
\end{theorem}
\end{thmbox}

\begin{proof}
部分積分によって
\begin{align*}
    \int_a^b f(t) \cdot \dot{\eta}(t) dt =
    [f(t) \cdot \eta(t)]_{t = a}^{t = b} - \int_a^b \dot{f}(t) \cdot \eta(t) dt
\end{align*}
である.
\(\eta \in X_0\)より\(\eta(a) = \eta(b) = 0\)であるから,右辺第\(1\)項は\(0\)である.
仮定~\ref{fundamental-lemma2-assumption}と\lemmaref{fundamental-lemma1}から任意の\(t \in [a, b]\)で\(\dot{f}(t) = 0\)となる.したがって\(f\)は定数関数である.
\end{proof}

\mathref{variational-method-frechet-derivative}の\(J'(q)\)は部分積分と\(\eta(a) = \eta(b) = 0\)によって次をみたすことがわかる.
\begin{align*}
    & J'(q)(\eta) \\
    & = \int_a^b D_1 f(q(t), \dot{q}(t)) \eta(t) dt + \int_a^ b D_2 f(q(t), \dot{q}(t))\dot{\eta}(t) dt \\
    & = \left[\left(\int_a^t D_1 f(q(\tau), \dot{q}(\tau)) d\tau\right) \eta(t)\right]_{t = a}^{t = b}
    - \int_a^b \left(\int_a^t D_1 f(q(\tau), \dot{q}(\tau)) d\tau \right) \dot{\eta}(\tau) dt \\
    & \mathbin{\hphantom{=}}{} + \int_a^b D_2f(q(t), \dot{q}(t)) \dot{\eta}(t) dt \\
    & = \int_a^b
    \left(
        \left(
        - \int_a^t D_1 f(q(\tau), \dot{q}(\tau)) d\tau
        +  D_2 f(q(t), \dot{q}(t))
        \right)
        \dot{\eta}(t)
    \right) dt
\end{align*}
\theoremref{fundamental-lemma2}より\(k_1, \ldots, k_n\)を定数として
\begin{align*}
    - \int_a^t D_1 f(q(\tau), \dot{q}(\tau)) d\tau
    + D_2 f(q(t), \dot{q}(t)) = \begin{pmatrix} k_1 & \cdots & k_n \end{pmatrix}
    \mathlabel{euler-lagrange-integral}
\end{align*}
が成り立つ.
\mathref{euler-lagrange-integral}をEuler--Lagrange方程式の積分形という.

\section{地球のモデル}

地球上の\(2\)地点\(\symup{P}, \symup{Q}\)を最短で結ぶ曲線を求める.
地中を掘って進むことは困難なので,それは不可能であるとする.
単純化のため地表の構造物や地球が回転楕円体であることは無視し,地球を球面\(S^2 = \{\tuple{x_1, x_2, x_3} \mid x_1^2 + x_2^2 + x_3^2 = 1\}\)でモデル化する.

計算をしやすくするため\(S^2\)の各点を球面座標系\(\xi = \tuple{\xi_0, \xi_1, \xi_2}\)で表す.
動径\(\xi_0\)は\(1\)に固定されるため,\(\xi_0\)を省略して\(\tuple{\xi_1, \xi_2}\)のように書く.
天頂角あるいは余緯度\(\xi_1\) \((0 < \xi_1 < \pi)\)は緯度に応じて
\begin{numcases}
    {\xi_1 =}
    \frac{90^\circ - \alpha^\circ}{180^\circ} \pi
        & \text{北緯\(\alpha^\circ\)のとき,} \nonumber \\[3pt]
    \frac{90^\circ + \alpha^\circ}{180^\circ} \pi
        & \text{南緯\(\alpha^\circ\)のとき} \nonumber
\end{numcases}
となる.
一方,方位角\(\xi_2 \) \((0 \leq \xi_2 < 2\pi)\)は
\begin{numcases}
    {\xi_2 =}
    \frac{\beta^\circ}{180^\circ} \pi
        & \text{東経\(\beta^\circ\)のとき,} \nonumber \\[3pt]
    \frac{360^\circ - \beta^\circ}{180^\circ} \pi
        & \text{西経\(\beta^\circ\)のとき} \nonumber
\end{numcases}
となる.
例えば北緯\(39^\circ\),東経\(141^\circ\)の地点は\(\tuple{\xi_1, \xi_2} = \tuple{\inlinefrac{17\pi}{60}, \inlinefrac{47\pi}{60}}\)と表される.
北極点\(\symup{N} = \tuple{0, 0, 1}\)と南極点\(\symup{S} = \tuple{0, 0, -1}\)は球面座標系では一意に表されないことに注意する.例えば\(\tuple{\xi_1, \xi_2} = \tuple{0, 0}, \tuple{0, \pi}\)はどちらも\(\symup{N}\)を表す.\(\symup{N}\)と\(\symup{S}\)を含む本初子午線は
\begin{align*}
    \Phi_0 = \left\{\tuple{x_1, x_2, x_3} \relmiddle{|} \text{\(\sqrt{x_1^2 + x_3^2} = 1\), \(x_1 > 0\), and \(x_2 = 0\)} \right\}
\end{align*}
と表すことができる.
\(S^2 \setminus \Phi_0\)の局所座標系として球面座標系\(\xi = \tuple{\xi_1, \xi_2}\)をとる.
\begin{gather*}
    \xi \colon
    \tuple{x_1, x_2, x_3}
    \mapsto
    \tuple{\xi_1, \xi_2}
    =
    \tuple{\arccos x_3, a(x_1, x_2)}, \\
    a(x_1, x_2) =
        \begin{cases}
            \atantwo(\inlinefrac{x_2}{x_1}) + \pi & \text{if \(x_1 < 0\)}, \\
            \atantwo(\inlinefrac{x_2}{x_1}) + 2 \pi & \text{if \(x_1 > 0\) and \(x_2 < 0\)}, \\
            \atantwo(\inlinefrac{x_2}{x_1}) & \text{otherwise},
        \end{cases} \\
    \atantwo(y, x) =
        \begin{cases}
            \pi & \text{if \(x_1 < 0\) and \(x_2 = 0\)}, \\[3pt]
            \displaystyle
            2\arctan\left(\frac{x_2}{\sqrt{x_1^2 + x_2^2} + x_1}\right) & \text{otherwise.}
        \end{cases}
\end{gather*}
写像\(\iota\)を\(S^2\)から\({\Real}^3\)への包含写像とする.
\(S^2 \setminus \Phi_0\)の点\(p = (p_1, p_2)\)については\(\iota\)を
\begin{align*}
    \iota\colon
    p
    \mapsto
    \begin{pmatrix}
        \iota_1(p) \\
        \iota_2(p) \\
        \iota_3(p)
    \end{pmatrix}
    =
    \begin{pmatrix}
        \sin p_1 \cos p_2 \\
        \sin p_1 \sin p_2 \\
        \cos p_1
    \end{pmatrix}
\end{align*}
と表すことができる.
微分\(d\iota\)は方向微分\(\partial_{\xi_1}, \partial_{\xi_2}\)を次のように写す.
\begin{gather*}
    d\iota (\partial_{\xi_1})
    = \sum_{i = 1}^3 \frac{\partial \iota_i}{\partial \xi_1}(p) \tpmbase{x_i}{\iota(p)}
    =
    \begin{pmatrix}
        \cos p_1 \cos p_2 \\
        \cos p_1 \sin p_2 \\
        -\sin p_1
    \end{pmatrix}, \\
    d\iota(\partial_{\xi_2})
    = \sum_{i = 1}^3 \frac{\partial \iota_i}{\partial \xi_2}(p) \tpmbase{x_i}{\iota(p)}
    =
    \begin{pmatrix}
        - \sin p_1 \sin p_1 \\
        \sin p_1 \cos p_2 \\
        0
    \end{pmatrix}.
\end{gather*}
ただし方向微分について,以下のような自然な同一視を行った.
\begin{align*}
    \tpmbase{x_1}{\iota(p)} = \begin{pmatrix}
        1 \\ 0 \\ 0
    \end{pmatrix},
    \quad
    \tpmbase{x_2}{\iota(p)} = \begin{pmatrix}
        0 \\ 1 \\ 0
    \end{pmatrix},
    \quad
    \tpmbase{x_3}{\iota(p)} = \begin{pmatrix}
        0 \\ 0 \\ 1
    \end{pmatrix}.
\end{align*}
ユークリッド空間\({\Real}^3\)の計量を\(g_\text{E}\)とし,
\(\iota\)によって\(S^2 \setminus \Phi_0\)に誘導される計量を\(g_p\)とする.
このとき
\begin{gather*}
    \begin{split}
    g_p(\partial_{\xi_1}, \partial_{\xi_2})
    =
    g_\text{E}(d\iota(\partial_{\xi_1}), d\iota(\partial_{\xi_1}))
    =
    \begin{pmatrix}
        \cos p_1 \cos p_2 \\
        \cos p_1 \sin p_2 \\
        - \sin p_1
    \end{pmatrix}
    \cdot
    \begin{pmatrix}
        \cos p_1 \cos p_2 \\
        \cos p_1 \sin p_2 \\
        - \sin p_1
    \end{pmatrix}
    = 1,
    \end{split} \\
    \begin{split}
    g_p(\partial_{\xi_1}, \partial_{\xi_2})
    =
    g_\text{E}(d\iota (\partial_{\xi_1}), d\iota (\partial_{\xi_2}))
    =
    \begin{pmatrix}
        \cos p_1 \cos p_2 \\
        \cos p_1 \sin p_2 \\
        - \sin p_1
    \end{pmatrix}
    \cdot
    \begin{pmatrix}
        - \sin p_1 \sin p_2 \\
        \sin p_1 \cos p_2 \\
        0
    \end{pmatrix}
    = 0,
    \end{split} \\
    \begin{split}
    g_p(\partial_{\xi_1}, \partial_{\xi_2})
    = g_\text{E}(d\iota (\partial_{\xi_2}), d\iota (\partial_{\xi_2}))
    =
    \begin{pmatrix}
        - \sin p_1 \sin p_2 \\
        \sin p_1 \cos p_2 \\
        0
    \end{pmatrix}
    \cdot
    \begin{pmatrix}
        - \sin p_1 \sin p_2 \\
        \sin p_1 \cos p_2 \\
        0
    \end{pmatrix}
    = \sin^2 p_1
    \end{split}
\end{gather*}
が成り立つ.したがって計量\(g_p\)に対応する行列\(G_p\)は
\begin{align*}
    G_p =
    \begin{pmatrix}
        g_p(\partial_{\xi_1}, \partial_{\xi_1}) & g_p(\partial_{\xi_1}, \partial_{\xi_2}) \\
        g_p(\partial_{\xi_2}, \partial_{\xi_1}) & g_p(\partial_{\xi_2}, \partial_{\xi_2})
    \end{pmatrix}
    =
    \begin{pmatrix}
        1 & 0 \\
        0 & \sin^2 p_1
    \end{pmatrix}
\end{align*}
となる.
よって\(p\)における速度ベクトル
\begin{align*}
    v = v_1 \tpmbase{\xi_1}{p} + v_2 \tpmbase{\xi_2}{p}
\end{align*}
について
\begin{align}
    g_p(v, v)
    &= v_1^2 g_p(\partial_{\xi_1}, \partial_{\xi_1}) + 2 v_1 v_2 g_p(\partial_{\xi_1}, \partial_{\xi_2}) + v_2^2 g_p(\partial_{\xi_2}, \partial_{\xi_2}) \notag \\
    &= \begin{pmatrix} v_1 & v_2 \end{pmatrix} G_p \begin{pmatrix} v_1 \\ v_2 \end{pmatrix} \notag \\
    &= v_1^2 + v_2^2 \sin^2 p_1 \mathlabel{great-circle-inner-product}
\end{align}
となる.

\section{地球上の2地点間の距離}
記号は前節と同じものを使う.
\(S^2 \setminus \Phi_0\)の異なる\(2\)点\(\symup{P},\symup{Q}\)をとり,
\(\symup{P}\)から\(\symup{Q}\)への正則な曲線を\(\gamma \colon [0, T] \to \openinterval{0, \pi} \times \openinterval{0, 2\pi}\)とし,\(q = \xi \circ \gamma\)とする.このとき\(q(t) = \coord{\theta(t), \varphi(t)}\)と表すことができる.
ここで\(\theta \colon [0, T] \to [0, \pi)\)と\(\varphi \colon [0, T] \to [0, 2 \pi)\)である.
曲線\(\gamma\)が正則であると仮定したため,暗に\(\theta, \varphi\)は\(C^1\)級であると仮定されていることに注意する.\(q\)の速度ベクトルは
\begin{align*}
    \dot{q}(t) = \dot{\theta}(t) \tpmbase{\xi_1}{q} + \dot{\varphi}(t) \tpmbase{\xi_2}{q}
\end{align*}
これは既に弧長による再パラメーター付けがされているものとする.
\mathref{great-circle-inner-product}より
関数\(f \colon \xi(S^2 \setminus \Phi_0) \times {\Real}^2 \to {\Real}\)を
\begin{gather*}
    f\colon (y, u) \mapsto \frac{u_1^2 + u_2^2 \sin^2 y_1}{2}
\end{gather*}
と定義すると,\(\gamma\)のエネルギーは
\begin{align}
    E(q) = \int_0^T f(q(t), \dot{q}(t)) dt
    = \int_0^T \frac{(\dot{\theta}(t))^2 + (\dot{\varphi}(t))^2 \sin^2 \theta(t)}{2} dt
    \mathlabel{great-circle-energy}
\end{align}
と表すことができる.\(f\)の\(\coord{q(t), \dot{q}(t)}\)における偏導関数は
\begin{gather*}
    D_1 f(q(t), v(t))
    =
    \begin{pmatrix}
        (\dot{\varphi}(t))^2 \sin \theta(t) \cos \theta(t)
        &
        0
    \end{pmatrix},
    \\
    D_2 f(q(t), v(t))
    =
    \begin{pmatrix}
        \dot{\theta}(t)
        &
        \dot{\varphi}(t) \sin^2 \theta(t)
    \end{pmatrix}
\end{gather*}
である.したがってこの問題のEuler--Lagrange方程式の積分形は
\begin{numcases}
    {}
    \dot{\theta}(t) = \int_0^t (\dot{\varphi}(\tau))^2 \sin \theta(\tau) \cos \theta(\tau) d\tau + k_1 \\
    \dot{\varphi}(t) \sin^2 \theta(t) = k_2
\end{numcases}
となる.
右辺の形から\(\dot{\theta}, \dot{\varphi}\)もまた\(C^1\)級である.
したがって両辺を\(t\)で微分して次の連立方程式を得る.
\begin{numcases}
    {}
    \ddot{\theta}(t) = \dot{\varphi}^2(t) \sin \theta(t) \cos \theta(t) \mathlabel{great-circle-euler-lagrange1} \\
    \ddot{\varphi}(t) = - 2 \frac{\dot{\theta}(t)}{\tan \theta(t)}\dot{\varphi}(t) \mathlabel{great-circle-euler-lagrange2}
\end{numcases}
エネルギー(\mathref{great-circle-energy})は\(\theta\)と\(\varphi\)の関係のみで定まる.
したがって\(\theta\)を\(\varphi\)の関数とみなす.
このとき\mathref{great-circle-euler-lagrange1}の左辺は
\begin{align}
    \ddot{\theta}
    &= \frac{d}{dt}\left(\frac{d\theta}{d\varphi}\frac{d\varphi}{dt}\right) \notag \\
    &= \left(\frac{d}{dt}\frac{d\theta}{d\varphi}\right)\frac{d\varphi}{dt}
        + \left(\frac{d\theta}{d\varphi}\right)\frac{d^2\varphi}{dt^2} \notag \\
    &= \left(\frac{d}{d\varphi}\frac{d\theta}{d\varphi}\frac{d\varphi}{dt}\right)\frac{d\varphi}{dt}
        + \left(\frac{d\theta}{d\varphi}\right)\frac{d^2\varphi}{dt^2} \notag \\
    &= \frac{d^2\theta}{d\varphi^2}\dot{\varphi}^2
        + \left(\frac{d\theta}{d\varphi}\right)\ddot{\varphi}. \mathlabel{great-circle-ddot-theta}
\end{align}
\mathref{great-circle-euler-lagrange2}について同様に
\begin{gather}
    \ddot{\varphi} = -2 \frac{\dot{\varphi}^2}{\tan \theta}\frac{d\theta}{d\varphi} \mathlabel{great-circle-ddot-varphi}
\end{gather}
\mathref{great-circle-ddot-varphi}を\mathref{great-circle-ddot-theta}にして
\begin{align*}
    \ddot{\theta} = \frac{d^2\theta}{d\varphi^2}\dot{\varphi}^2
        -2 \left(\frac{d\theta}{d\varphi}\right)^2 \frac{\dot{\varphi}^2}{\tan \theta}
\end{align*}
\mathref{great-circle-euler-lagrange1}に代入して
\begin{gather}
    \left(
        \frac{d^2\theta}{d\varphi^2}
        - \frac{2}{\tan \theta} \left(\frac{d\theta}{d\varphi}\right)^2
        - \sin \theta \cos \theta
    \right)
    \dot{\varphi}^2
    = 0 \mathlabel{great-circle-ode-product-form}.
\end{gather}
ここで
\begin{align*}
    \frac{d^2}{d \varphi^2}\frac{1}{\tan \theta}
    &= \frac{d}{d\varphi} \left(- \frac{1}{\sin^2 \theta} \frac{d\theta}{d\varphi} \right) \\
    &= 2 \frac{\cos \theta}{\sin^3 \theta}\left(\frac{d\theta}{d\varphi}\right)^2
        -\frac{1}{\sin^2 \theta} \frac{d^2\theta}{d\varphi^2} \\
    &= - \frac{1}{\sin^2 \theta}
        \left(
        \frac{d^2\theta}{d\varphi^2}
        -
        \frac{2}{\tan \theta}\left(\frac{d\theta}{d\varphi}\right)^2
    \right)
\end{align*}
を用いて\mathref{great-circle-ode-product-form}を書き換えると
\begin{align*}
    \left(
        \frac{d^2}{d\varphi^2}\frac{1}{\tan \theta}
        + \frac{1}{\tan \theta}
    \right)
    \dot{\varphi}^2
    = 0.
\end{align*}
したがって
\begin{numcases}
    {}
    \dot{\varphi} = 0 \\
    \frac{d^2}{d\varphi^2} \frac{1}{\tan \theta} + \frac{1}{\tan \theta} = 0 \mathlabel{great-circle-simple-harmonic-motion}
\end{numcases}
\mathref{great-circle-simple-harmonic-motion}は,\(y(\varphi) = \inlinefrac{1}{\tan \theta(\varphi)}\), \(D = \inlinefrac{d}{d\varphi}\)と置いて微分方程式
\begin{align}
    (D^2  + 1) y = 0 \label{great-circle-ode-operator}
\end{align}
に帰着される.
代数方程式\(\lambda^2 + 1 = 0\)の根が\(\lambda = \pm i\)であるから微分方程式\ref{great-circle-ode-operator}の解空間は
\begin{align*}
\Span \{\realpart e^{i\varphi}, \imaginarypart e^{i\varphi}, \realpart e^{-i\varphi}, \imaginarypart e^{-i\varphi}\} = \Span \{\cos \varphi, \sin \varphi\}
\end{align*}
に等しい.
したがって\(A, B\)を定数として
\begin{align*}
    y(\varphi) = \frac{1}{\tan \theta(\varphi)} = A \cos \varphi + B \sin \varphi
\end{align*}
と表すことができる.
この両辺に\(\sin \theta\)をかけると
\begin{align*}
    \cos \theta  = A \sin \theta \cos \varphi + B \sin \theta \sin \varphi.
\end{align*}
となる.
これは定ベクトル
\begin{align*}
    n_\bot = \begin{pmatrix} A \\ B \\ -1 \end{pmatrix}
\end{align*}
と曲線\(\iota \circ \gamma\)上の点が常に直交していること,すなわち\(\gamma\)の像が大円の一部であることを表す.
これで目的は達せられたが,せっかくなので\(A, B\)を具体的に求めておこう.
\(\symup{P} = \iota(\theta(0), \varphi(0)) = \coord{\symup{P}_1, \symup{P}_2, \symup{P}_3}, \symup{Q} = \iota(\theta(T), \varphi(T)) = \coord{\symup{Q}_1, \symup{Q}_2, \symup{Q}_3}\)とすると
\begin{numcases}
    {}
    \frac{1}{\tan \theta(0)} = A \cos \varphi(0) + B \sin \varphi(0) \nonumber \\
    \frac{1}{\tan \theta(T)} = A \cos \varphi(T) + B \sin \varphi(T) \nonumber.
\end{numcases}
これを解くと
\begin{align*}
    \begin{pmatrix}
        A \\
        B
    \end{pmatrix}
    &=
    \begin{pmatrix}
        \cos \varphi(0) & \sin \varphi(0) \\
        \cos \varphi(T) & \sin \varphi(T)
    \end{pmatrix}^{-1}
    \begin{pmatrix}
        \inlinefrac{1}{\tan \theta(0)} \\
        \inlinefrac{1}{\tan \theta(T)}
    \end{pmatrix} \\
    &=
    \frac{1}{\sin \theta(0) \sin \theta(T)\sin (\varphi(T) - \varphi(0))}
    \begin{pmatrix}
        \sin \varphi(T) & - \sin \varphi(0) \\
        - \cos \varphi(T) & \cos \varphi(0)
    \end{pmatrix}
    \begin{pmatrix}
        \cos \theta(0) \sin \theta(T) \\
        \cos \theta(T) \sin \theta(0)
    \end{pmatrix}
    \\
    &=
    \frac{1}{\symup{P}_x\symup{Q}_y - \symup{P}_y \symup{Q}_x}
    \begin{pmatrix}
        \symup{P}_z \symup{Q}_y - \symup{P}_y \symup{Q}_z \\
        \symup{P}_x \symup{Q}_z - \symup{P}_z \symup{Q}_x
    \end{pmatrix}.
\end{align*}
したがって
\begin{align*}
    n_\bot \mathbin{\|}
        \begin{pmatrix}
            \symup{P}_y \symup{Q}_z - \symup{P}_z \symup{Q}_y \\
            \symup{P}_z \symup{Q}_x - \symup{P}_x \symup{Q}_z \\
            \symup{P}_x\symup{Q}_y - \symup{P}_y \symup{Q}_x
        \end{pmatrix}
        = \overrightarrow{\symup{O}\symup{P}} \times \overrightarrow{\symup{O}\symup{Q}}.
\end{align*}
半径\(R\)の球面\(S^2(R)\)の\(2\)点\(\symup{P}, \symup{Q}\)の距離は,
\(\overrightarrow{\symup{O}\symup{P}}, \overrightarrow{\symup{O}\symup{Q}}\)
のなす角に等しい.
\begin{align*}
    R \arccos \left(%
        \frac{%
            \overrightarrow{\symup{O}\symup{P}}
        }{%
            \left\lVert \overrightarrow{\symup{O}\symup{P}} \right\rVert
        } \cdot
        \frac{%
            \overrightarrow{\symup{O}\symup{Q}}
        }{%
            \left\lVert \overrightarrow{\symup{O}\symup{Q}} \right\rVert
        }
    \right)
\end{align*}

\begin{figure}
\centering
\tdplotsetmaincoords{60}{130}
\begin{tikzpicture}[%
    scale=3.4,
    tdplot_main_coords
]
    \pgfmathsetmacro{\r}{1.0}

    \pgfmathsetmacro\panglephi{25}
    \pgfmathsetmacro\pangletheta{50}

    \pgfmathsetmacro\qanglephi{80}
    \pgfmathsetmacro\qangletheta{40}

    \coordinate (O) at (0, 0, 0);
    \node[left=2] at (0, 0, 0) {\(\origin\)};
    \node[above=5, right=2] at (0, 0, 1) {\(\symup{N}\)};
    \tdplotsetcoord{P}{\r}{\pangletheta}{\panglephi}
    \tdplotsetcoord{Q}{\r}{\qangletheta}{\qanglephi}

    % PROJECTIONS
    \draw[sLightGray, thin] plot[domain=0:90](
        {\r * sin(\x) * cos(\qanglephi)},
        {\r * sin(\x) * sin(\qanglephi)},
        {\r * cos(\x)}
    ) coordinate (projV);
    \draw[sLightGray, thin] plot[domain=0:90]({\r * cos(\x)}, {\r * sin(\x)}, 0);
    \draw[sLightGray, thin] (O)  -- ({\r * cos(\panglephi)}, {\r * sin(\panglephi)}, 0);
    \draw[sLightGray, thin] (O)  -- ({\r * cos(\qanglephi)}, {\r * sin(\qanglephi)}, 0);
    \tdplotdrawarc[sRed, -stealth]{(O)}{0.45 * \r}{0}{\qanglephi} {below=-1}{\(\varphi_2\)}
    \draw[sLightGray, thin] plot[domain=0:90](
        {\r * sin(\x) * cos(\panglephi)},
        {\r * sin(\x) * sin(\panglephi)},
        {\r * cos(\x)}
    ) coordinate (projP);

    \tdplotsetthetaplanecoords{\panglephi}
    \tdplotdrawarc[sBlue, -stealth, tdplot_rotated_coords]{(0, 0, 0)}{0.18 * \r}{0}{\pangletheta}{left=4, above=-1}{\(\theta_1\)}
    \tdplotsetthetaplanecoords{\qanglephi}
    \tdplotdrawarc[sRed, -stealth, tdplot_rotated_coords]{(0, 0, 0)}{0.22 * \r}{0}{\qangletheta}{right=3, above=-1}{\(\theta_2\)}

    \tdplotdrawarc[sBlue, -stealth]{(O)}{0.2 * \r}{0}{\panglephi} {below=-1}{\(\varphi_1\)}

    \draw[thick,-stealth] (0, 0, 0) -- (1.2 * \r, 0, 0) node[left=1, above=2]{};
    \draw[thick,-stealth] (0, 0, 0) -- (0, 1.2 * \r, 0) node[above=2]{};
    \draw[thick,-stealth] (0, 0, 0) -- (0,0, 1.2 * \r) node[below=3, left=1]{};

    \tdplotdefinepoints(0, 0, 0)%
        ({\r * sin(\pangletheta) * cos(\panglephi)}, {\r * sin(\pangletheta) * sin(\panglephi)}, {\r * cos(\pangletheta)})%
        ({\r * sin(\qangletheta) * cos(\qanglephi)}, {\r * sin(\qangletheta) * sin(\qanglephi)}, {\r * cos(\qangletheta)})
    \tdplotdrawpolytopearc[sGreen]{1}{anchor=west}{}

    \draw[sBlue, -stealth] (O)  -- (P) node[above=4, left=2] {\(\symup{P}\)};
    \draw[sRed, -stealth] (O)  -- (Q) node [below=10, left=-3] {\(\symup{Q}\)};

    \node[sGreen, right=40, above=-5] at ({\r * sin(30) * cos(50)}, {\r * sin(30) * sin(50)}, {\r * cos(30)}) {\(\displaystyle R \arccos \left(\frac{\overrightarrow{\symup{O}\symup{P}}}{\left\lVert \overrightarrow{\symup{O}\symup{P}} \right\rVert} \cdot \frac{\overrightarrow{\symup{O}\symup{Q}}}{\left\lVert \overrightarrow{\symup{O}\symup{Q}} \right\rVert}\right)\)};

\end{tikzpicture}
\end{figure}

\nocite{yajima}
\nocite{yamamoto}

\bibliographystyle{jecon}
\bibliography{references}

\end{document}
